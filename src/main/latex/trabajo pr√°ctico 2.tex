\documentclass[10pt,a4paper]{article}

\usepackage[spanish,activeacute,es-tabla]{babel}
\usepackage[utf8]{inputenc}
\usepackage{ifthen}
\usepackage{listings}
\usepackage{dsfont}
\usepackage{subcaption}
\usepackage{amsmath}
\usepackage[strict]{changepage}
\usepackage[top=1cm,bottom=2cm,left=1cm,right=1cm]{geometry}%
\usepackage{color}%
\newcommand{\tocarEspacios}{%
	\addtolength{\leftskip}{3em}%
	\setlength{\parindent}{0em}%
}

% Especificacion de procs

\newcommand{\In}{\textsf{in }}
\newcommand{\Out}{\textsf{out }}
\newcommand{\Inout}{\textsf{inout }}

\newcommand{\encabezadoDeProc}[4]{%
	% Ponemos la palabrita problema en tt
	%  \noindent%
	{\normalfont\bfseries\ttfamily proc}%
	% Ponemos el nombre del problema
	\ %
	{\normalfont\ttfamily #2}%
	\
	% Ponemos los parametros
	(#3)%
	\ifthenelse{\equal{#4}{}}{}{%
		% Por ultimo, va el tipo del resultado
		\ : #4}
}

\newenvironment{proc}[4][res]{%
	
	% El parametro 1 (opcional) es el nombre del resultado
	% El parametro 2 es el nombre del problema
	% El parametro 3 son los parametros
	% El parametro 4 es el tipo del resultado
	% Preambulo del ambiente problema
	% Tenemos que definir los comandos requiere, asegura, modifica y aux
	\newcommand{\requiere}[2][]{%
		{\normalfont\bfseries\ttfamily requiere}%
		\ifthenelse{\equal{##1}{}}{}{\ {\normalfont\ttfamily ##1} :}\ %
		\{\ensuremath{##2}\}%
		{\normalfont\bfseries\,\par}%
	}
	\newcommand{\asegura}[2][]{%
		{\normalfont\bfseries\ttfamily asegura}%
		\ifthenelse{\equal{##1}{}}{}{\ {\normalfont\ttfamily ##1} :}\
		\{\ensuremath{##2}\}%
		{\normalfont\bfseries\,\par}%
	}
	\renewcommand{\aux}[4]{%
		{\normalfont\bfseries\ttfamily aux\ }%
		{\normalfont\ttfamily ##1}%
		\ifthenelse{\equal{##2}{}}{}{\ (##2)}\ : ##3\, = \ensuremath{##4}%
		{\normalfont\bfseries\,;\par}%
	}
	\renewcommand{\pred}[3]{%
		{\normalfont\bfseries\ttfamily pred }%
		{\normalfont\ttfamily ##1}%
		\ifthenelse{\equal{##2}{}}{}{\ (##2) }%
		\{%
		\begin{adjustwidth}{+5em}{}
			\ensuremath{##3}
		\end{adjustwidth}
		\}%
		{\normalfont\bfseries\,\par}%
	}
	
	\newcommand{\res}{#1}
	\vspace{1ex}
	\noindent
	\encabezadoDeProc{#1}{#2}{#3}{#4}
	% Abrimos la llave
	\par%
	\tocarEspacios
}
{
	% Cerramos la llave
	\vspace{1ex}
}

\newcommand{\aux}[4]{%
	{\normalfont\bfseries\ttfamily\noindent aux\ }%
	{\normalfont\ttfamily #1}%
	\ifthenelse{\equal{#2}{}}{}{\ (#2)}\ : #3\, = \ensuremath{#4}%
	{\normalfont\bfseries\,;\par}%
}

\newcommand{\pred}[3]{%
	{\normalfont\bfseries\ttfamily\noindent pred }%
	{\normalfont\ttfamily #1}%
	\ifthenelse{\equal{#2}{}}{}{\ (#2) }%
	\{%
	\begin{adjustwidth}{+2em}{}
		\ensuremath{#3}
	\end{adjustwidth}
	\}%
	{\normalfont\bfseries\,\par}%
}

% Tipos

\newcommand{\nat}{\ensuremath{\mathds{N}}}
\newcommand{\ent}{\ensuremath{\mathds{Z}}}
\newcommand{\float}{\ensuremath{\mathds{R}}}
\newcommand{\bool}{\ensuremath{\mathsf{Bool}}}
\newcommand{\cha}{\ensuremath{\mathsf{Char}}}
\newcommand{\str}{\ensuremath{\mathsf{String}}}

% Logica

\newcommand{\True}{\ensuremath{\mathrm{true}}}
\newcommand{\False}{\ensuremath{\mathrm{false}}}
\newcommand{\Then}{\ensuremath{\rightarrow}}
\newcommand{\Iff}{\ensuremath{\leftrightarrow}}
\newcommand{\implica}{\ensuremath{\longrightarrow}}
\newcommand{\IfThenElse}[3]{\ensuremath{\mathsf{if}\ #1\ \mathsf{then}\ #2\ \mathsf{else}\ #3\ \mathsf{fi}}}
\newcommand{\yLuego}{\land _L}
\newcommand{\oLuego}{\lor _L}
\newcommand{\implicaLuego}{\implica _L}

\newcommand{\cuantificador}[5]{%
	\ensuremath{(#2 #3: #4)\ (%
		\ifthenelse{\equal{#1}{unalinea}}{
			#5
		}{
			$ % exiting math mode
			\begin{adjustwidth}{+2em}{}
				$#5$%
			\end{adjustwidth}%
			$ % entering math mode
		}
		)}
}

\newcommand{\existe}[4][]{%
	\cuantificador{#1}{\exists}{#2}{#3}{#4}
}
\newcommand{\paraTodo}[4][]{%
	\cuantificador{#1}{\forall}{#2}{#3}{#4}
}

%listas

\newcommand{\TLista}[1]{\ensuremath{seq \langle #1\rangle}}
\newcommand{\lvacia}{\ensuremath{[\ ]}}
\newcommand{\lv}{\ensuremath{[\ ]}}
\newcommand{\longitud}[1]{\ensuremath{|#1|}}
\newcommand{\cons}[1]{\ensuremath{\mathsf{addFirst}}(#1)}
\newcommand{\indice}[1]{\ensuremath{\mathsf{indice}}(#1)}
\newcommand{\conc}[1]{\ensuremath{\mathsf{concat}}(#1)}
\newcommand{\cab}[1]{\ensuremath{\mathsf{head}}(#1)}
\newcommand{\cola}[1]{\ensuremath{\mathsf{tail}}(#1)}
\newcommand{\sub}[1]{\ensuremath{\mathsf{subseq}}(#1)}
\newcommand{\en}[1]{\ensuremath{\mathsf{en}}(#1)}
\newcommand{\cuenta}[2]{\mathsf{cuenta}\ensuremath{(#1, #2)}}
\newcommand{\suma}[1]{\mathsf{suma}(#1)}
\newcommand{\twodots}{\ensuremath{\mathrm{..}}}
\newcommand{\masmas}{\ensuremath{++}}
\newcommand{\matriz}[1]{\TLista{\TLista{#1}}}
\newcommand{\seqchar}{\TLista{\cha}}

\renewcommand{\lstlistingname}{Código}
\lstset{% general command to set parameter(s)
	language=Java,
	morekeywords={endif, endwhile, skip},
	basewidth={0.47em,0.40em},
	columns=fixed, fontadjust, resetmargins, xrightmargin=5pt, xleftmargin=15pt,
	flexiblecolumns=false, tabsize=4, breaklines, breakatwhitespace=false, extendedchars=true,
	numbers=left, numberstyle=\tiny, stepnumber=1, numbersep=9pt,
	frame=l, framesep=3pt,
	captionpos=b,
}

\usepackage{caratula} % Version modificada para usar las macros de algo1 de ~> https://github.com/bcardiff/dc-tex


\titulo{Trabajo Práctico 2}
\subtitulo{Diseño e implementación de estructuras}

\fecha{\today}

\materia{Algoritmos y Estructuras de Datos}
\grupo{GRUPO EPICO - LSSPMSSLDBLMOLTIKQGP}

\integrante{Condori, Alex}{163/23}{nocwe11@gmail.com}
\integrante{Della Rosa, Facundo}{1317/23}{dellarosafacundo@gmail.com}
\integrante{García, Lucía}{706/22}{luciiamar19@gmail.com}
\integrante{Ser, Gonzalo Nicolás}{573/21}{ser.gonzalo10@gmail.com}


\begin{document}

\maketitle

\section{Invariantes de Representación de las clases de Java}



\subsection{Trie}

\begin{itemize}
	\item Todos los nodos tienen un array -hijos- de 256 posiciones que representan
	      el código ASCII de cada caracter.
	\item Los atributos palabra, padre y valor del nodo raíz son Null.
	\item Agregar una clave que ya está en el Trie pisará su valor.
	\item Los hijos de un nodo no pueden tener caminos que apunten a distinto nodo pero de igual carácter.
	      Es decir, cada hijo conduce a una única palabra.
	\item Los nodos que no son la raíz tienen un puntero a su nodo padre.
	\item El tamaño del Trie siempre es igual a la cantidad de claves.
	\item Intentar obtener una clave que no esté en el Trie retornará Null.
	\item Cada clave está formada por el recorrido de los nodos correspondientes a sus caracteres.
	\item Para cada nodo, su array de hijos tienen valores nulos en las posiciones que no llevan a una clave.
	\item En el último nodo del recorrido de la clave se encuentra el valor y un String -palabra-
	      correspondiente a dicha clave.
	\item Cada nodo es alcanzable, es decir que se puede acceder a todos los nodos que estén en el Trie.
\end{itemize}

\subsection{DatosMateria}

\begin{itemize}
	\item La cantidad de estudiantes es siempre $\geq$ 0.
	\item La longitud de la lista de libretas es igual a la cantidad de estudiantes.
	\item La longitud de cada elemento de la lista de libretas tiene como máximo 7 caracteres (xxxx/xx).
	\item La longitud del Array “Docentes por cargo” es 4.
	\item Toda posición en el Array “Docentes por cargo” es un entero positivo.
	\item El cupo disponible es igual al mínimo entre:
	      250 * cantidad de Profesores, 100 * cantidad de JTPs, 20 * cantidad de Ayudantes de 1° y
	      30 * cantidad de Ayudantes de 2° menos la cantidad de estudiantes.
	\item Cada elemento de la lista “Nombres materia” contiene un puntero al Trie de materias
	      de una carrera y el nombre que tiene la materia en dicha carrera.
	\item Cumple el Invariante de Representación de \bfseries{ParRefCarreraMateria}.
\end{itemize}

\subsection{ParRefCarreraMateria}

\begin{itemize}
	\item Ref Carrera contiene una referencia a un Trie de materias de una carrera que pertenece
	      al Trie Carreras del Sistema SIU.
	\item Materia es el nombre que tiene una materia de la carrera referenciada por RefCarrera.
	      La materia pertenece al Trie de materias de dicha carrera.
\end{itemize}

\subsection{ListaEnlazada}

\begin{itemize}
	\item Todos los nodos, exceptuando al Primero, son accesibles a través de su anterior.
	\item Para 2 nodos a y b cualesquiera, si el siguiente de a es b, entonces
	      el anterior de b es a.
	\item El atributo Size es igual la cantidad de nodos.
	\item El puntero siguiente al último nodo apunta a null.
	\item El puntero anterior al primer nodo apunta a null.
	\item Cada nodo tiene 2 punteros que apuntan hacia él.
\end{itemize}

\subsection{SistemaSiu}

\begin{itemize}
	\item el Trie Estudiantes contiene las libretas universitarias de todos los estudiantes
	      inscriptos en el sistema. El valor de sus elementos es la cantidad de materias en las que
	      está inscripto el estudiante con esa libreta.
	\item Cada valor del trie “estudiantes” tiene como máximo 7 caracteres (xxxx/xx).
	\item El Trie Carreras contiene los nombres de todas las carreras que están en el sistema.
	      El valor de sus elementos es una referencia al Trie de materias.
	\item El Trie de materias contiene todas las materias de una carrera que están en el sistema.
	      El valor de sus elementos es del tipo DatosMateria.
	\item Cada Trie cumple la invariante de representación del Trie escrito anteriormente.
	\item Cada ListaEnlazada cumple la invariante de representación de listaEnlazada escrito anteriormente.
	\item Cada DatosMateria cumple la invariante de representación de DatosMateria escrito anteriormente.
\end{itemize}


\section{Aclaraciones}

\subsection{complejidad temporal del constructor del SistemaSIU}

El constructor del SistemSIU va recorriendo las carreras y materias en un orden distinto al que está
en el enunciado. El algoritmo va recorriendo el Array de infoMaterias (el primer for loop).
Cada elemento de este Array contiene a su vez un Array con paresCarreraMateria (el segundo for loop).
\smallskip
Cada parCarreraMateria tiene un String materia y un String carrera.
Primero, o bien recorremos o bien creamos y agregamos la carrera al Trie Carreras, según corresponda.
En total esto se realiza tantas veces como materias agreguemos a esa carrera.
\begin{equation}
	O(\sum_{c\in C}^{}\mid c\mid * \mid M_{c}\mid )
\end{equation}
Segundo, dentro del Trie de materias de esa carrera, agregamos la materia del par.
Esta operación en total la realizamos tantas veces como materias vayamos a agregar (contando todos sus nombres).\\
\begin{equation}
	O(\sum_{m\in M}^{}\sum_{n\in N_{m}}^{}\mid n\mid)
\end{equation}
Por último, simplemente recorremos la lista de Estudiantes y agregamos cada String a una lista enlazada.
\begin{equation}
	O(E)
\end{equation}
En conjunto, las complejidades se suman y cumplen con lo pedido en el enunciado.
\end{document}
